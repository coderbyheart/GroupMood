\section{Fazit}

Die Umsetzung erfüllt die in Abschnitt~\ref{l:gruppenbefragung} beschriebenen Anforderung. Meetings können serverseitig mithilfe eines einfachen Admin-Interfaces und weiteren Hilfsmitteln, z.B. einem Wizard zum Import von Vorlesungsfolien, beschrieben werden und anschließend komfortabel in der App dargestellt werden. Das Interface in der App ist dabei leicht verständlich und flexibel genug gestaltet, um verschiedene Arten von Meetings abbilden zu können -- für die Funktionen zur Foto-Abstimmung wurden lediglich neue GUI-Elemente zum Anlegen von Veranstaltungen und Themen erforderlich, ansonsten wird die Darstellung verwendet, wie sie z.B. bei Vorlesungen zum Einsatz kommen würde. In der App stehen Wertebereichs-, Einfach- und Mehrfachauswahlfragen zur Verfügung, die entweder einmalig oder kontinuierlich beantwortet werden können. \emph{GroupMood} ist in der vorgestellten Version somit bereits in der Lage viele Anwendungsfälle abzudecken.

\bigskip

\emph{GroupMood} findet sich \href{https://market.android.com/details?id=de.hsrm.mi.mobcomp.y2k11grp04}{im Android-Market}, der Quellcode des Projekts steht auf \href{https://github.com/tacker/GroupMood}{GitHub} zur Verfügung.